\chapter{Технологический раздел}

\section{Выбор языка и среды программирования}

В качестве языка программирования был выбран язык C. Данный ЯП используется для разработки всех модулей ядра и драйверов операционной системы Linux.

Среда программирования --- Microsoft Visual Studio Code \cite{vscode}. Данное решение является кроссплатформенным и бесплатным, а также имеет открытый исходный код.

\section{Точки входа в модуль}
В листингах \ref{lst:md_init}--\ref{lst:md_exit} представлены точки входа в модуль --- init и exit --- в которых происходит создание и уничтожение очереди работ, а также регистрация и дерегистрация обработчика прерываний от клавиатуры. 

\begin{lstlisting}[caption={Точки входа в модуль (часть 1)} \label{lst:md_init}]
static int __init md_init(void)
{
    int ret;
    printk(KERN_INFO "Module Initialization.\n");
    ret = request_irq(KB_IRQ, (irq_handler_t)kb_irq_handler, IRQF_SHARED, "Custom Handler", (void *)(kb_irq_handler));
    if (ret != 0)
        printk(KERN_INFO "Cannot Request IRQ for keyboard.\n");

    work = kmalloc(sizeof(struct work_struct), GFP_KERNEL);
    if (work)
    {
        INIT_WORK((struct work_struct *)work, audio_controll);
    }
    wq = create_workqueue("keyboard_queue");

    printk(KERN_INFO "Module Inited.\n");
    return 0;
}
\end{lstlisting}

\clearpage

\begin{lstlisting}[caption={Точки входа в модуль (часть 2)} \label{lst:md_exit}]
static void __exit md_exit(void)
{
    free_irq(KB_IRQ, (void *)(kb_irq_handler));
    flush_workqueue(wq);
    destroy_workqueue(wq);

    if (work)
        kfree(work);

    printk(KERN_INFO "Module Exit.");
}

module_init(md_init);
module_exit(md_exit);
\end{lstlisting}

\section{Обработка прерываний клавиатуры}
В листингах \ref{lst:kb_irq_handler}--\ref{lst:system_volume_control3} представлен обработчик прерываний клавиатуры.

\begin{lstlisting}[caption={Обработчик прерываний клавиатуры} \label{lst:kb_irq_handler}]
irq_handler_t kb_irq_handler(int irq, void *dev_id, struct pt_regs *regs)
{
    spin_lock(&k_lock);
    scancode = inb(0x60);
    spin_unlock(&k_lock);

    queue_work(wq, work);

    return (irq_handler_t)IRQ_HANDLED;
}
\end{lstlisting}

\begin{lstlisting}[caption={Обработчик нижней половины прерывания(часть 1)} \label{lst:system_volume_control1}]
void audio_controll(struct work_struct *work)
{
    static int alt = 0;
    static long audio_flag = -1;

    spin_lock(&k_lock);
    switch (scancode)
    {
    case ALT_PRESSED:
        alt = 1;
        break;
\end{lstlisting}

\clearpage

\begin{lstlisting}[caption={Обработчик нижней половины прерывания(часть 2)} \label{lst:system_volume_control2}]
    case ALT_UNPRESSED:
        alt = 0;
        break;
    case LEVEL0_PRESSED:
        if (alt)
            audio_flag = 0;
        break;
    case LEVEL1_PRESSED:
        if (alt)
            audio_flag = 10;
        break;
    case LEVEL2_PRESSED:
        if (alt)
            audio_flag = 20;
        break;
    case LEVEL3_PRESSED:
        if (alt)
            audio_flag = 30;
        break;
    case LEVEL4_PRESSED:
        if (alt)
            audio_flag = 40;
        break;
    case LEVEL5_PRESSED:
        if (alt)
            audio_flag = 50;
        break;
    case LEVEL6_PRESSED:
        if (alt)
            audio_flag = 60;
        break;
    case LEVEL7_PRESSED:
        if (alt)
            audio_flag = 70;
        break;
    case LEVEL8_PRESSED:
        if (alt)
            audio_flag = 80;
        break;
    case LEVEL9_PRESSED:
        if (alt)
            audio_flag = 90;
        break;
\end{lstlisting}

\clearpage

\begin{lstlisting}[caption={Обработчик нижней половины прерывания(часть 3)} \label{lst:system_volume_control3}]
    case LEVELMINUS_PRESSED:
        if (alt && current_audio > 0)
            audio_flag = current_audio - 1;
        break;
    case LEVELPLUS_PRESSED:
        if (alt && current_audio < 100)
            audio_flag = current_audio + 1;
        break;
    default:
        break;
    }

    if (audio_flag >= 0)
    {
        printk(KERN_INFO "System Volume Changing Started.\n");
        current_audio = audio_flag;
        char buffer[10];
        sprintf(buffer, "%d", current_audio);
        char *argv[] = {"/home/leerycorsair/os_course/audio_controller", buffer, NULL};
        static char *envp[] = {
            "HOME=/",
            "TERM=linux",
            "PATH=/sbin:/bin:/usr/sbin:/usr/bin", NULL};

        call_usermodehelper(argv[0], argv, envp, UMH_WAIT_PROC);
        printk(KERN_INFO "System Volume Changed to %d\%.\n", current_audio);
    }

    audio_flag = -1;
    spin_unlock(&k_lock);
}
\end{lstlisting}

\section{Функция изменения громкости}

В листинге \ref{lst:system_volume_controller} приведена структура, описывающая звуковое устройство для которого создается функция изменения громкости.

\clearpage

\begin{lstlisting}[caption={Структура system\_volume\_controller} \label{lst:system_volume_controller}]
struct audio_controller
{
    long min, max;
    snd_mixer_t *handle;
    snd_mixer_selem_id_t *sid;
    const char *card;
    const char *selem_name;
} a_controller;
\end{lstlisting}

В данной структуре созданы следующие поля:
\begin{itemize}
    \item min и max --- минимальный и максимальный физический уровень громкости;
    \item handle --- микшерное устройство;
    \item sid --- идентификатор микшерного элемента;
    \item card --- имя устройства воспроизведения;
    \item selem\_name --- имя микшерного элемента.
\end{itemize}

В листингах \ref{lst:system_volume_change1}--\ref{lst:system_volume_change2} представлена функция изменения уровня громкости.

\begin{lstlisting}[caption={Функция изменения громкости (часть 1)} \label{lst:system_volume_change1}]
void audio_level_change(long audio_level)
{
    a_controller.card = "default";
    a_controller.selem_name = "Master";

    if (snd_mixer_open(&(a_controller.handle), 0) < 0)
        return;

    if (snd_mixer_attach(a_controller.handle, a_controller.card) < 0)
        return;

    if (snd_mixer_selem_register(a_controller.handle, NULL, NULL) < 0)
        return;

    if (snd_mixer_load(a_controller.handle) < 0)
        return;

    snd_mixer_selem_id_alloca(&(a_controller.sid));
    snd_mixer_selem_id_set_index(a_controller.sid, 0);
    snd_mixer_selem_id_set_name(a_controller.sid, a_controller.selem_name);
\end{lstlisting}

\clearpage

\begin{lstlisting}[caption={Функция изменения громкости (часть 2)} \label{lst:system_volume_change2}]
    snd_mixer_elem_t *elem = snd_mixer_find_selem(a_controller.handle, a_controller.sid);
    if (elem == NULL)
        return;

    snd_mixer_selem_get_playback_volume_range(elem, &(a_controller.min), &(a_controller.max));
    if (snd_mixer_selem_set_playback_volume_all(elem, audio_level * a_controller.max / 100) < 0)
        return;

    snd_mixer_close(a_controller.handle);
}
\end{lstlisting}

\section{Сборка разработанного модуля}

Для компиляции загружаемого модуля используется утилита make. Конфигурационный файл makefile приведен в листинге \ref{lst:makefile} \cite{makefile}.

\begin{lstlisting}[caption={makefile для сборки} \label{lst:makefile}]
CONFIG_MODULE_SIG=n
CURRENT = $(shell uname -r)
KDIR = /lib/modules/$(CURRENT)/build
PWD = $(shell pwd)
TARGET = audio_module
obj-m := $(TARGET).o


default: audio_controller
	$(MAKE) -C$(KDIR) M=$(PWD) modules
  
audio_controller: audio_controller.c
	gcc -o $@ $@.c -lasound
\end{lstlisting}